\chapter{Aldagai anitzeko funtzioak}
\section{Aldagai anitzeko funtzioen jarraitutasuna}
\underline{10.2.ariketa} Aztertu funtzio honen jarraitasuna jatorrian.
$$f(x,y) = \left\{ \begin{array}{cl}
		x\sin \dfrac{1}{y} +y \sin \dfrac{1}{x},  &  x \neq 0 \, \mbox{ eta } \, y \neq 0, \\
                 0,	                                &  x=0 \mbox{ edo } y=0.
		   \end{array} \right. $$
		   
		 
Funtzio bat jatorrian jarraia izateko honako baldintza hau bete behar du:

\begin{equation*}
    \text{Funtzioa jarraia jatorrian}
    \Longleftrightarrow
    \boxed{\lim_{x,y \to (0,0)}f(x,y)=f(0,0)=0}
\end{equation*}
	   
Baldintza hau betetzen den edo ez jakiteko funtzioaren limitea jatorrian kalkulatu behar dugu. Horretarako, honako pausu hauek jarraituko ditugu:
\begin{enumerate}
    \item Dimentsio bakarreko limitea
    \begin{eqnarray*}
        &g(y)&=\lim_{x \to 0}f(x,y)=\lim_{x \to 0} x\sin \dfrac{1}{y} +y \sin \dfrac{1}{x}= \nexists\\
        &h(x)&=\lim_{y \to 0}f(x,y)=\lim_{y \to 0} x\sin \dfrac{1}{y} +y \sin \dfrac{1}{x}= \nexists
    \end{eqnarray*}
    
    \item Limite berrituak
    \begin{eqnarray*}
        &&\lim_{y \to 0}g(y)=\nexists\\
        &&\lim_{x \to 0}h(x)=\nexists
    \end{eqnarray*}
    
    Limite berrituak ez dira existitzen.Hortaz, ezin dugu 1.9.teorema aplikatu eta ez dugu informaziorik lortu. 
    
    Jarraian, norabide bakarreko limiteekin probatuko dugu. Horrela ikusiko dugu norabide ezberdinez jatorrira hurbiltzen bagara ea limitea 0 den.
    \item Norabide bakarreko limiteak \newline
    \newline
    $\displaystyle{ \lim_{ \begin{array}{c} \scriptstyle (x,y) \rightarrow (0,0) \\ \scriptstyle y=mx \\ \end{array} } f(x,y)= \lim_{x \rightarrow 0} x\sin \dfrac{1}{mx} +mx \sin \dfrac{1}{x}=\lim_{x \rightarrow 0} x\cdot\left(\sin \dfrac{1}{mx} +m \sin \dfrac{1}{x}\right)=\nexists } $.
    \newline
    $\displaystyle{ \lim_{ \begin{array}{c} \scriptstyle (x,y) \rightarrow (0,0) \\ \scriptstyle y=\lambda x^2 \\ \end{array} }f(x,y)= \lim_{x \rightarrow 0} x\sin \dfrac{1}{\lambda x^2} +\lambda x^2 \sin \dfrac{1}{x}=\lim_{x \rightarrow 0} x\cdot\left(\sin \dfrac{1}{\lambda x^2} +\lambda x \sin \dfrac{1}{x}\right)=\nexists } $.
    
    Norabide bakarreko limiteak existitzen ez direnez ez digute informazio gehiago eman.
    
    Beraz, azken aukera limiteen definizioa apikatzea da. 
    \item Definizioa aplikatu
    \begin{align*}
    &\lim_{x,y \to (0,0)}f(x,y)=0\\
    &\forall \epsilon >0 \quad \exists \delta(\epsilon)>0 \quad / \quad0<||(x,y)||<\delta \rightarrow ||f(x,y)||<\epsilon  \\
    \\
    &\text{Hau da frogatu behar dugun baldintza.}\\
    &\forall \epsilon >0 \quad \exists \delta(\epsilon)>0 \quad / \quad0<||(x,y)||<\delta \rightarrow \left|x\cdot \sin \frac{1}{y}+ y \sin \frac{1}{x}\right|<\epsilon  \\
    &\text{ Hasteko azken desberdintzaren ezkerreko atala hartu eta sinplifikatzen saiatuko gara.}\\
    &\left|x\cdot \sin \frac{1}{y}+ y \sin \frac{1}{x}\right|\leq\left| x\right|\cdot\left |\sin\frac{1}{y}\right |+\left|y\right|\cdot\left|\sin\frac{1}{x}\right|\leq \abs{x}+\abs{y}\\
    &\text{Aurreko desberdintzaren eskuineko atalean lortu duguna ikusita,}\\
    &\text{aukerarik honena baturaren norma erabiltzea da:} \;\; \| (x,y) \| =\abs{x}+\abs{y}\ \\
    &\| (x,y)\| < \delta \quad \text{bada,} \quad | x | + | y | < \delta \quad \text{dugu.}\\
    &\text{Hortik lortu dugun}\quad\left|x\cdot \sin \frac{1}{y}+ y \sin \frac{1}{x}\right|\leq \abs{x}+\abs{y}\quad \text{desberdintzaz baliatuz}\\
    &\left|x\cdot \sin \frac{1}{y}+ y \sin \frac{1}{x}\right|< \delta\quad \text{izango dugu.}\\
    &\text{Ondorioz nahikoa da}\quad \delta < \epsilon \quad \text{izatea }\\
    &\left|x\cdot \sin \frac{1}{y}+ y \sin \frac{1}{x}\right| < \epsilon \quad \text{ere beteko delako.}\\
    &\text{Hau guztiarekin limite bikoitza 0 dela frogatu dugu} \quad \boxed{\lim_{x,y \to (0,0)}f(x,y)=0}
    \end{align*}
\end{enumerate}

\begin{align*}
    &\text{Alde batetik,}\lim_{x,y \to (0,0)}f(x,y)=0\quad \text{frogatu dugu, bestetik,}\quad f(0,0)=0 \quad \text{da.}\\ &\text{Ondorioz} \quad \boxed{\lim_{x,y \to (0,0)}f(x,y)=f(0,0)=0} \quad \text{beteko da eta funtzioa jarraia da jatorrian.}
\end{align*}










\chapter{Aldagai anitzeko funtzioen diferentziagarritasuna}
\section{Berretura-seriezko garapena}
\underline{25.4.ariketa} Kalkulatu ezazu funtzioaren Taylorren garapena ematen den puntuaren inguruan.
\begin{equation*}
    f(x,y)=cos(x)cos(y), \quad (0,\frac{\pi}{2}) \text{ puntuan eta 3.ordenaraino.}
\end{equation*}
\begin{equation*}
\begin{split}
    &\text{Tayloren garapenaren formula 3.ordeneraino:}\\
    &f(x,y)=f(a,b)+\frac{1}{1!}(D_1f(a,b)(x-a)+D_2f(a,b)(y-b))+\\
    &\frac{1}{2!}(D_{11}f(a,b)(x-a)^2 +2 D_{12}f(a,b)(x-a)(y-b)+D_{22}f(a,b)(y-b)^2)+\\
    &\frac{1}{3|}(D_{111}f(a,b)(x-a)^3+3D_{112}f(a,b)(x-a)^2(y-b)+3D_{122}f(a,b)(x-a)(y-b)^2+\\
    &D_{222}f(a,b)(y-b)^3)\\
\end{split}
\end{equation*}

Lehenengo eta behin, formulan ordezkatu behar ditugun gai guztiak kalkulatuko ditugu.
\begin{equation*}
    \begin{array}{lll}
    & f(x,y)=\cos{x}+\cos{y}\qquad &f(a,b)=f(0,\frac{\pi}{2})=\cos{0}\cos{\frac{\pi}{2}}=0\\
    &(a,b)=(0,\frac{\pi}{2})&\\
    \\
    & D_1f(x,y)=-\sin{x}\cos{y}\qquad & D_1f(0,\frac{\pi}{2})=-\sin{0}\cos{\frac{\pi}{2}}=0 \\
    & D_2f(x,y)=-\cos{x}\sin{y}& D_2f(0,\frac{\pi}{2})=-\cos{0}\sin{\frac{\pi}{2}=-1}\\
    \\
    & D_{11}f(x,y)=-\cos{x}\cos{y}& D_{11}f(0,\frac{\pi}{2})=-\cos(0)\cos{\frac{\pi}{2}}=0\\
    
    & D_{12}f(x,y)=\sin{x}\sin{y} &D_{12}f(0,\frac{\pi}{2})=\sin{0}\sin{\frac{\pi}{2}}=0\\
    
    &D_{22}f(x,y)=-\cos{x}\cos{y}&D_{22}f(0,\frac{\pi}{2})=-\cos{0}\cos{\frac{\pi}{2}}=0\\
    \\
    &D_{111}f(x,y)=-\sin{x}\cos{y}&D_{111}f(0,\frac{\pi}{2})=-\sin{0}\cos{\frac{\pi}{2}}=0\\
    
    &D_{112}f(x,y)=\cos{x}\sin{y}&D_{112}f(0,\frac{\pi}{2})=\cos{0}\sin{\frac{\pi}{2}}=1\\
    
    &D_{122}f(x,y)=\sin{x}\cos{y}&D_{122}f(0,\frac{\pi}{2})=\sin{0}\cos{\frac{\pi}{2}}=0\\
    
    &D_{222}f(x,y)=\cos{x}\sin{y}&D_{222}f(0,\frac{\pi}{2})=\cos{0}\sin{\frac{\pi}{2}}=1\\
    \end{array}
\end{equation*}
Ondoren, 0 atera zaizkigun gai guztiak formulatik ken ditzakegu.
\begin{equation*}
\begin{split}
    &f(x,y)=\cancel{f(a,b)}+\frac{1}{1!}(\cancel{D_1f(a,b)(x-a)}+D_2f(a,b)(y-b))+\\
    &\frac{1}{2!}(\cancel{D_{11}f(a,b)(x-a)^2} +2 \cancel{D_{12}f(a,b)(x-a)(y-b)}+\cancel{D_{22}f(a,b)(y-b)^2)}+\\
    &\frac{1}{3|}(\cancel{D_{111}f(a,b)(x-a)^3}+3D_{112}f(a,b)(x-a)^2(y-b)+\cancel{3D_{122}f(a,b)(x-a)(y-b)^2}+\\
    &D_{222}f(a,b)(y-b)^3)\\
    &f(x,y)=D_2f(a,b)\left(y-b\right)+\frac{1}{3!}\left(3D_{112}f(a,b)(x-a)^2(y-b)+ D_{222}f(a,b)(y-b)^3 \right)\\
    &\\
\end{split}
\end{equation*}
Azkenik goian kalkulatutako gaiak formulan ordezkatuko ditugu.
\begin{align*}
    &f(x,y)=D_2f(0,\frac{\pi}{2})\left(y-\frac{\pi}{2}\right)+\frac{1}{3!}\left(3D_{112}f(0,\frac{\pi}{2})(x-0)^2(y-\frac{\pi}{2})+ D_{222}f(0,\frac{\pi}{2})(y-\frac{\pi}{3})^3 \right)=\\
    &-\left(y-\frac{\pi}{2}\right)+\frac{1}{6}\left(3x^2(y-\frac{\pi}{2})+(y-\frac{\pi}{2})^3\right)=-\left(y-\frac{\pi}{2}\right)+\frac{1}{2}x^2\left(y-\frac{\pi}{2}\right)+\frac{1}{6}\left(y-\frac{\pi}{2}\right)^3
\end{align*}


Beraz, ona hemen emaitza:
\begin{equation*}
    \boxed{f(x,y)=-\left(y-\frac{\pi}{2}\right)+\frac{1}{2}x^2\left(y-\frac{\pi}{2}\right)+\frac{1}{6}\left(y-\frac{\pi}{2}\right)^3}
\end{equation*}

Wolfram Mathematica programa erabiliz irudikatzen baldin badugu:
\begin{itemize}
    \item Gorriz taylorren garapena 3.ordeneraino
    \item Urdinez funtzioa
    \item Berdez puntua 
\end{itemize}

\begin{figure*}[h]
    \includegraphics[scale=0.5]{2.ariketa Taylor-1.png}
    \includegraphics[scale=0.5]{2.ariketa Taylor-2.png}
    \label{fig:my_label}
\end{figure*}

\chapter{Aldagai anitzeko funtzioen analisi lokala}
\section{Aldagai anitzeko funtzioen mutur baldintzatuak}
\underline{19.ariketa} Zer luze dira azalera txikiena duen V bolumeneko paralelepipedoaren ertzak? Eta azalera handienekoarenak?

\begin{equation*}
    V = xyz
\end{equation*}
\begin{equation*}
    A(X,Y,Z) = 2xy + 2xz + 2zy = 2(xy+xz+xy)
\end{equation*}
\begin{equation*}
    F(x,y,z) = (2xy + 2xz + 2zy) + \lambda(xyz-V)
\end{equation*}

\begin{equation*}
\left.
  \begin{tabular}{ c }
    $D_{1f}(x,y,z)=2(y+z) + \lambda(yz)$ \\
    $D_{2f}(x,y,z)=2(x+z) + \lambda(xz)$ \\
    $D_{3f}(x,y,z)=2(x+y) + \lambda(yx)$ \\
    $V = xyz$ 
    \end{tabular}
  \right\}
  \rightarrow
  \left.
  \begin{tabular}{ c }
    $2(y+z) + \lambda(yz) = 0$ \\
    $2(x+z) + \lambda(xz) = 0$ \\
    $2(x+y) + \lambda(yx) = 0$ \\
    $V = xyz$ 
  \end{tabular}
  \right\}
  \rightarrow
    \left.
  \begin{tabular}{ c }
    $\lambda = \frac{-2(y+z)}{yz}$ \\
    $\lambda = \frac{-2(x+z)}{xz}$ \\
    $\lambda = \frac{-2(x+y)}{yx}$ \\
  \end{tabular}
  \right\}
  \rightarrow
\end{equation*}
\begin{equation*}
\rightarrow
  \left\{
  \begin{tabular}{ c }
     $\frac{-2(y+z)}{yz} = \frac{-2(x+z)}{xz} \rightarrow (y+z)xz=(x+z)yz \rightarrow z(yx+zx-yx-zy)=0$\\
     $\frac{-2(y+z)}{yz} = \frac{-2(x+y)}{xy} \rightarrow (y+z)xy=(x+y)yz \rightarrow xy^2 + xyz = xyz -y^2z$
  \end{tabular}
  \right.
  \rightarrow
\end{equation*}
\begin{equation*}
\rightarrow
  \left\{
  \begin{tabular}{ c }
    $z(zx-zy)=0 \rightarrow z^2(x-y)=0 \rightarrow \cancel{z=0} \text{ edo } x=y$ \\
    $xy^2 + xyz -xyz +y^2z=0 \rightarrow y^2(x+z)=0 \rightarrow \cancel{y=0} \text{ edo } x=-z$ \\
  \end{tabular}
  \right.
  \rightarrow
\end{equation*}
\begin{equation*}
 \rightarrow
 \left.
 \begin{tabular}{ c }
    $x=y=-z$ \\
    $V=xyz$ \\
  \end{tabular}
  \right\}
  \rightarrow
   \boxed{P(\sqrt[3]{V}, \sqrt[3]{V}, \sqrt[3]{V})}
\end{equation*}
\newpage
\begin{equation*}
    \boxed{g(x,y) = F\Bigg(x,y,\frac{V}{xy}\Bigg)} = 2xy + 2x\Bigg(\frac{V}{xy}\Bigg) + 2y\Bigg(\frac{V}{xy}\Bigg)= 2xy + \frac{2V}{y} + \frac{2V}{x}
\end{equation*}
Segida Hesetarra lortu
\begin{itemize}
    \item $\lambda_1$ lortu
        \begin{equation*}
        D_{1g}(x,y) = \frac{-2V}{x^2} + 2y
        \rightarrow
        D_{11g}(x,y) = \frac{4V}{x^3}
        \rightarrow
        \lambda_{1} = D_{11g}(\sqrt[3]{V}, \sqrt[3]{V}) = 4
        \end{equation*}
\item $\lambda_2$ lortu
    \begin{equation*}
        D_{1g}(x,y) = \frac{-2V}{x^2} + 2y
        \rightarrow
        \left\{
        \begin{tabular}{c}
        $D_{11g}(x,y) = \frac{4V}{x^3}$ \\
        $D_{21g}(x,y) = 2$
        \end{tabular}
        \right.
    \end{equation*}
    \begin{equation*}
        D_{2g}(x,y) = \frac{-2V}{y^2} + 2x
        \rightarrow
        \left\{
        \begin{tabular}{c}
        $D_{12g}(x,y) = 2$ \\
        $D_{21g}(x,y) = \frac{4V}{x^3}$
        \end{tabular}
        \right.
    \end{equation*}
    \begin{equation*}
        \lambda_{2} = 
    \begin{vmatrix}
        D_{11g} & D_{12g}\\
        D_{21g} & D_{22g}
    \end{vmatrix}
    =
    \begin{vmatrix}
        \frac{4V}{x^3} & 2\\
        2 & \frac{4V}{x^3}
    \end{vmatrix}
    \end{equation*}
    \begin{equation*}
        \lambda_{2}(\sqrt[3]{V},\sqrt[3]{V}) = 
    \begin{vmatrix}
        4 & 2\\
        2 & 4
    \end{vmatrix}
    = 12
    \end{equation*}
\item Segida hesetarra
    \begin{equation*}
    \{1, \lambda_{1}, \lambda_{2}\} = \{1, 4, 12\}
    \rightarrow
    \boxed{\text{Minimoa }\Big(\sqrt[3]{V}, \sqrt[3]{V}, \sqrt[3]{V}, 6\cdot(\sqrt[3]{V})^2\Big)}
    \end{equation*}
\end{itemize}





\chapter{Integral mugagabea}
\underline{7.27.ariketa} Kalkulatu ondoko integral hau
\begin{equation*}
    \int \frac{\sqrt{x}-\sqrt[6]{x}}{\sqrt[3]{x}+1}\ dx;
\end{equation*}
(TODO: Azaldu)
Integrala berehalakoa ez denez integrazio metodo bat erabili behar dugu, kasu honetan aldagai-aldaketa erabiliko dugu. $t=\sqrt[6]{x}$ eta $dt=dx$ ordezkapenak egingo ditugu.
\begin{equation*}
    \begin{split}
    &\int \frac{\sqrt{x}-\sqrt[6]{x}}{\sqrt[3]{x}+1}\ dx = \int \frac{t^3-t}{t^2+1}6t^5dt = 6 \cdot \int \frac{t^8-t^6}{t^2+1}dt =
     6 \cdot \int \frac{(t^2+1)(t^6-2t^9+2t^2-2)+2}{(t^2+1)}dt = & 
     \cr &6 \cdot \int t^6-2t^4+2t^2-2+\frac{2}{t^2+1}dt = &
     \end{split}
\end{equation*}
Integrala sinplifikatu ondoren berehalakoa bihurtzen da.
\begin{equation*}
     \begin{split}
         &6 \cdot                  [\frac{t^7}{7}-\frac{2t^5}-{5}+\frac{2t^3}{3}-2t+2\int \frac{1}{t^2+1}dt] = &
         \cr & \frac{6}{7} \cdot t^7 - \frac{12}{5}t^5+4t^3-12t+12\arctan{t} + k = &
         \cr &\frac{6}{7} \sqrt[6]{x^7}-\frac{12}{5}\sqrt[6]{x^5}+4+\sqrt{x}-12\cdot\sqrt[6]{x}+12\arctan{(\sqrt[6]{x})}+k&
    \end{split}
\end{equation*}




\chapter{Integral mugatua}
\section{Integral mugatuen aplikazioak}
\underline{13.7.ariketa} Kalkulatu ondoko funtzioak osatzen duen gorputzaren bolumena

\begin{equation*}
    x^2+(y-b)^2=a^2,\ \text{torua}\ b>a \ \text{izanik;}
    \end{equation*}
    \newline
    Ikusten dugunez, gorputz honek eragiten duen bolumena (kalkulatu behar dugun bolumena) hau izango litzateke:

    
   \begin{figure*}[h]
    
 \includegraphics[scale=1]{4.png}
 \includegraphics[scale=1.2]{4b.png}
    \end{figure*}
    \newpage
    
    
Horretarako formula hau erabiliko dugu bolumena kalkulatzeko:\\

$$V=\int_{-a}^{a} f^2(x)-g^2(x) dx$$
$$V= \pi \int_{-a}^{a} (b+\sqrt{a^2-x^2})^2-(b-\sqrt{a^2-x^2})^2 dx$$


    
    
    
    
    
    
    
    
