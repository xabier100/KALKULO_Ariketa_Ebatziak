\chapter{Aldagai anitzeko funtzioak}
\section{Aldagai anitzeko funtzioen jarraitutasuna}
\underline{10.2.ariketa} Aztertu funtzio honen jarraitasuna
$$f(x,y) = \left\{ \begin{array}{cl}
		x\sin \dfrac{1}{y} +y \sin \dfrac{1}{x},  &  x \neq 0 \, \mbox{ eta } \, y \neq 0, \\
                 0,	                                &  x=0 \mbox{ edo } y=0.
		   \end{array} \right. $$
Funtizoa jarraia jatorrian \Longleftrightarrow
\[
\lim_{x,y \to (0,0)}f(x,y)=f(0,0)=0
\]

\begin{itemize}
    \item Dimetsio bakarreko limitea
    \begin{eqnarray*}
        &g(y)&=\lim_{x \to 0}f(x,y)=\lim_{x \to 0} x\sin \dfrac{1}{y} +y \sin \dfrac{1}{x}= \nexists\\
        &h(x)&=\lim_{y \to 0}f(x,y)=\lim_{y \to 0} x\sin \dfrac{1}{y} +y \sin \dfrac{1}{x}= \nexists
    \end{eqnarray*}
    
    \item Limite berrituak
    \begin{eqnarray*}
        &&\lim_{y \to 0}g(y)=\nexists\\
        &&\lim_{x \to 0}h(x)=\nexists
    \end{eqnarray*}
    
    \item Norabide bakarreko limiteak \newline
    \newline
    $\displaystyle{ \lim_{ \begin{array}{c} \scriptstyle (x,y) \rightarrow (0,0) \\ \scriptstyle y=mx \\ \end{array} } f(x,y)= \lim_{x \rightarrow 0} x\sin \dfrac{1}{mx} +mx \sin \dfrac{1}{x}=\lim_{x \rightarrow 0} x\cdot\left(\sin \dfrac{1}{mx} +m \sin \dfrac{1}{x}\right)=\nexists } $.
    \newline
    $\displaystyle{ \lim_{ \begin{array}{c} \scriptstyle (x,y) \rightarrow (0,0) \\ \scriptstyle y=\lambda x^2 \\ \end{array} }f(x,y)= \lim_{x \rightarrow 0} x\sin \dfrac{1}{\lambda x^2} +\lambda x^2 \sin \dfrac{1}{x}=\lim_{x \rightarrow 0} x\cdot\left(\sin \dfrac{1}{\lambda x^2} +\lambda x \sin \dfrac{1}{x}\right)=\nexists } $.
    \newpage
    \item Definizioa aplikatu
    \begin{align*}
    &\lim_{x,y \to (0,0)}f(x,y)=0?\\
    &\forall \epsilon >0 \quad \exists \delta(\epsilon)>0 \quad / \quad0<||(x,y)||<\delta \rightarrow ||f(x,y)||<\epsilon  \\
    &\left|x\cdot \sin \frac{1}{y}+ y \sin \frac{1}{x}\right|\leq\left| x\right|\cdot\left |\sin\frac{1}{y}\right |+\left|y\right|\cdot\left|\sin\frac{1}{x}\right|\leq \abs{x}+\abs{y}\leq \delta < \epsilon\\
    \end{align*}
\end{itemize}

Beraz $\underline{\lim_{x,y \to (0,0)}f(x,y)=f(0,0)=0}$ baldintza betetzen denez funtzioa jarraitua da.




\chapter{Aldagai anitzeko funtzioen diferentziagarritasuna}
\section{Berretura-seriezko garapena}
\underline{25.4.ariketa} Kalkulatu zazu funtzioaren Taylorren garapena ematen den puntuaren inguruan
\begin{equation*}
    f(x,y)=cos(x)cos(y), \quad (0,\frac{\pi}{2}) \text{ puntuan eta 3.ordenaraino.}
\end{equation*}
\\
Tayloren garapenaren formula 3.ordeneraino:
\begin{equation*}
\begin{split}
    &f(x,y)=f(a,b)+\frac{1}{1!}(D_1f(a,b)(x-a)+D_2f(a,b)(y-b))+\\
    &\frac{1}{2!}(D_{11}f(a,b)(x-a)^2 +2 D_{12}f(a,b)(x-a)(y-b)+D_{22}f(a,b)(y-b)^2)+\\
    &\frac{1}{3|}(D_{111}f(a,b)(x-a)^3+3D_{112}f(a,b)(x-a)^2(y-b)+3D_{122}f(a,b)(x-a)(y-b)^2+\\
    &D_{222}f(a,b)(y-b)^3)
\end{split}
\end{equation*}
\begin{equation*}
    \begin{array}{lll}
    & f(x,y)=\cos{x}+\cos{y}\qquad &f(a,b)=f(0,\frac{\pi}{2})=\cos{0}\cos{\frac{\pi}{2}}=0\\
    &(a,b)=(0,\frac{\pi}{2})&\\
    \\
    & D_1f(x,y)=-\sin{x}\cos{y}\qquad & D_1f(0,\frac{\pi}{2})=-\sin{0}\cos{\frac{\pi}{2}}=0 \\
    & D_2f(x,y)=-\cos{x}\sin{y}& D_2f(0,\frac{\pi}{2})=-\cos{0}\sin{\frac{\pi}{2}=-1}\\
    \\
    & D_{11}f(x,y)=-\cos{x}\cos{y}& D_{11}f(0,\frac{\pi}{2})=-\cos(0)\cos{\frac{\pi}{2}}=0\\
    
    & D_{12}f(x,y)=\sin{x}\sin{y} &D_{12}f(0,\frac{\pi}{2})=\sin{0}\sin{\frac{\pi}{2}}=0\\
    
    &D_{22}f(x,y)=-\cos{x}\cos{y}&D_{22}f(0,\frac{\pi}{2})=-\cos{0}\cos{\frac{\pi}{2}}=0\\
    \\
    &D_{111}f(x,y)=-\sin{x}\cos{y}&D_{111}f(0,\frac{\pi}{2})=-\sin{0}\cos{\frac{\pi}{2}}=0\\
    
    &D_{112}f(x,y)=\cos{x}\sin{y}&D_{112}f(0,\frac{\pi}{2})=\cos{0}\sin{\frac{\pi}{2}}=1\\
    
    &D_{122}f(x,y)=\sin{x}\cos{y}&D_{122}f(0,\frac{\pi}{2})=\sin{0}\cos{\frac{\pi}{2}}=0\\
    
    &D_{222}f(x,y)=\cos{x}\sin{y}&D_{222}f(0,\frac{\pi}{2})=\cos{0}\sin{\frac{\pi}{2}}=1\\
    \end{array}
\end{equation*}
Beraz gure taylorren garpenaren formulatik 0 atera zaizkigun gai guztiak ken ditzakegu
\begin{equation*}
\begin{split}
    &f(x,y)=\cancel{f(a,b)}+\frac{1}{1!}(\cancel{D_1f(a,b)(x-a)}+D_2f(a,b)(y-b))+\\
    &\frac{1}{2!}(\cancel{D_{11}f(a,b)(x-a)^2} +2 \cancel{D_{12}f(a,b)(x-a)(y-b)}+\cancel{D_{22}f(a,b)(y-b)^2)}+\\
    &\frac{1}{3|}(\cancel{D_{111}f(a,b)(x-a)^3}+3D_{112}f(a,b)(x-a)^2(y-b)+\cancel{3D_{122}f(a,b)(x-a)(y-b)^2}+\\
    &D_{222}f(a,b)(y-b)^3)
\end{split}
\end{equation*}

Goian kalkulatutako gaiak formulan ordezkatuz
\begin{align*}
    &f(x,y)=D_2f(0,\frac{\pi}{2})\left(y-\frac{\pi}{2}\right)+\frac{1}{3!}\left(3D_{112}f(0,\frac{\pi}{2})(x-0)^2(y-\frac{\pi}{2})+ D_{222}f(0,\frac{\pi}{2})(y-\frac{\pi}{3})^3 \right)=\\
    &-\left(y-\frac{\pi}{2}\right)+\frac{1}{6}\left(3x^2(y-\frac{\pi}{2})+(y-\frac{\pi}{2})^3\right)=-\left(y-\frac{\pi}{2}\right)+\frac{1}{2}x^2\left(y-\frac{\pi}{2}\right)+\frac{1}{6}\left(y-\frac{\pi}{2}\right)^3
\end{align*}

Beraz, ona hemen emaitza:
\begin{equation*}
    \boxed{f(x,y)=-\left(y-\frac{\pi}{2}\right)+\frac{1}{2}x^2\left(y-\frac{\pi}{2}\right)+\frac{1}{6}\left(y-\frac{\pi}{2}\right)^3}
\end{equation*}

\chapter{Aldagai anitzeko funtzioen analisi lokala}
\section{Aldagai anitzeko funtzioen mutur baldintzatuak}
\underline{19.ariketa} Zer luze dira azalera txikiena duen V bolumeneko paralelepipedoaren ertzak? Eta azalera handienekoarenak?



\chapter{Integral mugagabea}
\underline{7.27.ariketa} Kalkulatu ondoko integral hau
\begin{equation*}
    \int \frac{\sqrt{x}-\sqrt[6]{x}}{\sqrt[3]{x}+1}\ dx;
\end{equation*}



\chapter{Integral mugatua}
\section{Integral mugatuen aplikazioak}
\underline{13.7.ariketa} Kalkulatu ondoko funtzioak osatzen duen gorputzaren bolumena

\begin{equation*}
    x^2+(y-b)^2=a^2,\ \text{torua}\ b>a \ \text{izanik;}
\end{equation*}