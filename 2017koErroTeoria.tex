%Erro honek gai guztietarako balio du


\documentclass[12pt,epsf]{basque-book}
%%%-\documentclass[12pt,epsf]{liburu}-%%%
%%%-\documentclass[12pt,epsf]{libro}-%%%

\usepackage{cancel}
\usepackage{amsmath, amssymb, latexsym}
%arydshln, 
\usepackage{epsf, graphics, graphicx}
\usepackage{fancyhdr}
\usepackage{fncychap}
\usepackage[latin1]{inputenc}
%\usepackage{babel}
\usepackage{makeidx}
\usepackage{natbib}
\usepackage{psfrag}
%\usepackage[spanish]{babel}
%\usepackage{textcomp}
\usepackage{theorem}
%%%-\usepackage{wrapfig}-%%%
%%%-\usepackage[T1]{fontenc}-%%%
%\usepackage[utf8]{inputenc}
\usepackage[basque]{babel}


\makeindex
\oddsidemargin 0.5cm
\evensidemargin -0.5cm
\textwidth 16cm
\textheight 23.5cm
\topmargin 0cm
\footskip 1cm
\allowdisplaybreaks

\pagestyle{fancy}
\renewcommand{\contentsname}{Aurkibidea}
\providecommand{\abs}[1]{\lvert#1\rvert}
\renewcommand{\chaptermark}[1]{ \markboth{ \thechapter \ #1 }{}}
%\renewcommand{\chaptermark}[1]{ \markboth{ #1 \ \thechapter }{} }
\renewcommand{\thepage}{\roman{page}}
%\chapterstyle{change}
\fancyhf{} 
\fancyhead[LE,RO]{\small\slshape\thepage}
\fancyhead[RE]{\small\slshape \nouppercase{\leftmark}}
\fancyhead[LO]{\small\slshape \nouppercase{\rightmark}}


%%%%%%%%%%%%%%%%%%
%%%-\setlength{\headheight}{14pt}-%%%
%\setcounter{secnumdepth}{3}
%%%%%%%%%%%%%%%%%%


\newcommand{\NN}{\mathbb N}
\newcommand{\ZZ}{\mathbb Z}
\newcommand{\QQ}{\mathbb Q}
\newcommand{\II}{\mathbb I}
\newcommand{\RR}{\mathbb R}
\newcommand{\CC}{\mathbb C}

\newcommand{\kala}{\mathcal A}
\newcommand{\kalb}{\mathcal B}
%\newcommand{\calb}{\mathcal B}
\newcommand{\kalc}{\mathcal C}
%\newcommand{\ce}{\mbox{${\cal C}$}}
\newcommand{\kale}{\mathcal E}
\newcommand{\ff}{\mathbb F}
%\newcommand{\efe}{\mathbb F}
%\newcommand{\FF}{\mathbb F}
\newcommand{\kalf}{\mathcal F}
%\newcommand{\fe}{\mbox{${\mathcal F}$}}
\newcommand{\kalg}{\mathcal G}
%\newcommand{\mge}{\mbox{${\cal G}$}}
\newcommand{\kalh}{\mathcal H}
\newcommand{\kk}{\mathbb K}
%\newcommand{\KK}{\mathbb K}
\newcommand{\kall}{\mathcal L}
\newcommand{\kalm}{\mathcal M}
%\newcommand{\calm}{\mathcal M}
\newcommand{\kalp}{\mathcal P}
%\newcommand{\pe}{\mbox{${\mathcal P}$}}
\newcommand{\kalr}{\mathcal R}
%\newcommand{\calr}{\mathcal R}
\newcommand{\rel}{\mathcal R}
%\newcommand{\rel}{\mbox{${\mathcal R}$}}
\newcommand{\kals}{\mathcal S}
%\newcommand{\ese}{\mbox{${\cal S}$}}
\newcommand{\kalt}{\mathcal T}
%\newcommand{\te}{\mbox{${\cal T}$}}
\newcommand{\kalv}{\mathcal V}
%\newcommand{\calv}{\mathcal V}


\newcommand{\aalj}{\mbox{{\rm anizalj }}}
\newcommand{\ageo}{\mbox{{\rm anizgeo }}}
\newcommand{\antz}{\ensuremath{\stackrel{a}{\sim}}}
\newcommand{\Ao}{\stackrel{\circ}{A}}
\newcommand{\arccot}{\mbox{{\rm arccot }}}
\newcommand{\Arg}{\mbox{{\rm Arg}}}
\newcommand{\bek}[1]{\mbox{\textbf{\textit{#1}}}}
\newcommand{\bin}[2]{\left( \begin{array}{@{}c@{}}
			          #1 \\ #2
			    \end{array} \right) }
\newcommand{\Bo}{\stackrel{\circ}{B}}
%\newcommand{\diag}{\mbox{{\rm Diag }}}
%\newcommand{\ed}{\ensuremath{\stackrel{d}{\sim}}} 
%\newcommand{\ef}{\ensuremath{\stackrel{e.f.}{\sim}}} 
%\newcommand{\eh}{\ensuremath{\stackrel{\ese}{\sim}}} 
\newcommand{\Eo}{\stackrel{\circ}{E}}
%\newcommand{\ep}{\ensuremath{\stackrel{\te}{\sim}}} 
%\newcommand{\eps}{\ensuremath{\varepsilon}}
\newcommand{\epsi}{\ensuremath{\varepsilon}}
\newcommand{\err}{\ensuremath{\stackrel{err}{\sim}}}
%\newcommand{\erre}{\ensuremath{\stackrel{r}{\sim}}} 
\newcommand{\ext}{\mbox{{\rm ext}}}
%\newcommand{\gammas}{\ensuremath{\stackrel{\Gamma_S}{\sim}}}
\newcommand{\gl}{\mbox{{\rm GL}}}
\newcommand{\im}{\mbox{{\rm Im }}}
\newcommand{\ind}{\mbox{{\rm ind }}}
\newcommand{\ir}{\mbox{{\rm Im }}}
%\newcommand{\im}{\mbox{{\rm Im }}}
\newcommand{\kar}{\mbox{{\rm card }}}
\newcommand{\Log}{\mbox{{\rm Log }}}
%%%-KENDU-\newcommand{\max}{\mbox{{\rm max }}}
%\newcommand{\mcd}{\mbox{{\rm m.c.d. }}}
%\newcommand{\mcd}{\mbox{{\rm mcd }}}
%\newcommand{\mcd}{\mbox{{\rm zkh }}}
%\newcommand{\mcm}{\mbox{{\rm m.c.m. }}}
%\newcommand{\mcm}{\mbox{{\rm mcm }}}
%\newcommand{\mcm}{\mbox{{\rm mkt }}}
%%%-KENDU-\newcommand{\min}{\mbox{{\rm min }}}
%\newcommand{\mini}{\mbox{{\rm min }}}
\newcommand{\mkt}{\mbox{{\rm mkt }}}
%\newcommand{\mod}{\mbox{{\rm mod }}}
\newcommand{\NNo}{\stackrel{\circ}{\NN}}
\newcommand{\norma}[1]{\parallel #1 \parallel}
\newcommand{\nukleo}{\mbox{{\rm Ker }}}
%\newcommand{\nucleo}{\mbox{{\rm Ker }}}
%\newcommand{\pf}{\ensuremath{\stackrel{PF}{\sim}}} 
%\newcommand{\pt}{\ensuremath{\stackrel{PT}{\sim}}}
\newcommand{\rang}{\mbox{{\rm rang }}}
%\newcommand{\rank}{\mbox{{\rm rang }}}
%\newcommand{\rank}{\mbox{{\rm rank }}}
\newcommand{\re}{\mbox{{\rm Re }}}
%\newcommand{\refc}[1]{(\ref{#1})}
%\newcommand{\se}{\ensuremath{\stackrel{s}{\sim}}}
%\newcommand{\si}{\ensuremath{\stackrel{s}{\sim}}}
\newcommand{\sig}{\mbox{{\rm sig}}}
\newcommand{\zkh}{\mbox{{\rm zkh}}}
\newcommand{\zn}{\mbox{{\rm zeinu}}}
\newcommand{\zut}{\ensuremath{\stackrel{zut}{\sim}}}
%\newcommand{\¿}{\mbox{?`}}


%\def\alg{\mathop{\rm aniz alj}\nolimits}
%\def\geo{\mathop{\rm aniz geo}\nolimits}
%\def\kar{\mathop{\rm card}\nolimits}
%\def\card{\mathop{\rm card}\nolimits}
%\def\Im{\mathop{\rm Im}\nolimits}
%\def\ind{\mathop{\rm ind}\nolimits}
%\def\log{\mathop{\rm log}\nolimits}
%\def\mcd{\mathop{\rm mcd}\nolimits}
%\def\mcm{\mathop{\rm mcm}\nolimits}
%\def\mini{\mathop{\rm mín}\nolimits}
%\def\mkt{\mathop{\rm mkt}\nolimits}
%\def\sig{\mathop{\rm sig}\nolimits}
%\def\zkh{\mathop{\rm zkh}\nolimits}



\newcommand{\ebazpen}{\noindent{\bf Ebazpena.}}
\newcommand{\erantzun}{\noindent{\bf Erantzuna.}}
\newcommand{\froga}{\noindent{\bf Froga.}}
\newcommand{\hitzarmen}{\noindent{\bf Hitzarmena.}}
\newcommand{\idazkera}{\noindent{\bf Idazkera.}}
\newcommand{\notacion}{\noindent{\bf Notazioa. }}
\newcommand{\ohar}{\noindent{\bf Oharra.}}
\newcommand{\oharrak}{\noindent{\bf Oharrak.}}



\theoremstyle{change}
%\newtheorem{theorem}{Teorema.}[chapter]
\newtheorem{teorema}{Teorema}[chapter]
%\newtheorem{theorem}{Teorema.}[section]
%\newtheorem{teorema}{Teorema.}[section]

%\newtheorem{ex}[theorem]{Ejemplo}
%\newtheorem{ex}[theorem]{Adibidea.}
\newtheorem{adibide}[teorema]{Adibidea.}
%\newtheorem{example}[theorem]{Adibidea.}
%\newtheorem{exs}[theorem]{Adibideak.}
%\newtheorem{examples}[theorem]{Adibideak.}
\newtheorem{adibideak}[teorema]{Adibideak.}

%\newtheorem{conjetura}[theorem]{Conjetura}
%\newtheorem{aieru}[theorem]{Aierua.}
\newtheorem{aieru}[teorema]{Aierua.}

%\newtheorem{definition}[theorem]{Definición}
%\newtheorem{definition}[theorem]{Definizioa.}
\newtheorem{definizio}[teorema]{Definizioa.}

%\newtheorem{corollary}[theorem]{Corolario}
%\newtheorem{corollarydefinition}[theorem]{Corolario-Definición}
%\newtheorem{corollary}[theorem]{Korolarioa.}
\newtheorem{korolario}[teorema]{Korolarioa.}

%\newtheorem{lemma}[theorem]{Lema.}
\newtheorem{lema}[teorema]{Lema.}

%\newtheorem{rem}[theorem]{Observación}
%\newtheorem{obs}[theorem]{Oharra.}
%\newtheorem{rem}[theorem]{Oharra.}
%\newtheorem{remarks}[theorem]{Oharrak.}
%\newtheorem{ohar}[teorema]{Oharra.}
%\newtheorem{oharrak}[teorema]{Oharrak.}

%\newtheorem{consecuencia}[theorem]{Ondorioa.}
%\newtheorem{cons}[theorem]{Ondorioak.}
%\newtheorem{ondorio}[theorem]{Ondorioa.}
\newtheorem{ondorio}[teorema]{Ondorioa.}
\newtheorem{ondorioak}[teorema]{Ondorioak.}

%\newtheorem{iprinciple}[theorem]{Indukzio-printzipioa.}
%\newtheorem{principle}[theorem]{Indukzio-printzipioa.}
%\newtheorem{oprinciple}[theorem]{Ordezkapen-printzipioa.}
%\newtheorem{principle}[theorem]{Ordezkapen-printzipioa.}
\newtheorem{printzipio}[teorema]{Printzipioa.}

%\newtheorem{problem}{Problema.}
\newtheorem{problema}[teorema]{Problema.}

%\newtheorem{property}[theorem]{Propietatea.}
\newtheorem{propietate}[teorema]{Propietatea.}
%\newtheorem{ppts}[theorem]{Propietateak.}
%\newtheorem{properties}[theorem]{Propietateak.}
\newtheorem{propietateak}[teorema]{Propietateak.}

%\newtheorem{proposition}[theorem]{Proposici\'on}
%\newtheorem{proposition}[theorem]{Proposizioa.}
\newtheorem{proposizio}[teorema]{Proposizioa.}


%\newenvironment{ejemplo}{\begin{example} \em } {\end{example}}
%\newenvironment{example}{\begin{ex} \rm} {\end{ex}}
\newenvironment{adi}{\begin{adibide} \em } {\end{adibide}}
%\newenvironment{ejemplos}{\begin{examples} \em 
%                          \begin{description} \item \end{description}
%                          \begin{enumerate} \end{enumerate}}
%                         {\end{examples}}
%\newenvironment{examples}{\begin{exs} \rm
%                          \begin{description} \item \end{description}}
%                         {\end{exs}}
%\newenvironment{examples}{\begin{exs} \rm
%                          \begin{description} \item \end{description} 
%                          \begin{enumerate} \end{enumerate}}
%                         {\end{exs}}
\newenvironment{adiitem}{\begin{adibideak} \em 
                           \begin{description} \item \end{description}}
                          {\end{adibideak}}
%\newenvironment{adibideak}{\begin{adibideak} \em 
%                           \begin{description} \item \end{description}
%                           \begin{enumerate} \end{enumerate}}
%                          {\end{adibideak}}

%\newenvironment{observacion}{\begin{obs} \em } {\end{obs}}
%\newenvironment{observacion}{\begin{rem} \em } {\end{rem}}
%\newenvironment{oharra}{\begin{obs} \em } {\end{obs}}
%\newenvironment{oharra}{\begin{rem} \em } {\end{rem}}
%\newenvironment{oharra}{\begin{ohar} \em } {\end{ohar}}
%\newenvironment{observaciones}{\begin{remarks} \em 
%                               \begin{description} \item \end{description}
%                               \begin{enumerate} \end{enumerate}}
%                              {\end{remarks}}
%\newenvironment{oharrak}{\begin{remarks} \em
%                         \begin{description} \item \end{description}
%                         \begin{enumerate} \end{enumerate}}
%                        {\end{remarks}}
%\newenvironment{oharitem}{\begin{oharrak} \em 
%                         \begin{description} \item \end{description}
%                         \begin{enumerate} \end{enumerate}}
%                         {\end{oharrak}}

%\newenvironment{consecuencias}{\begin{cons} \em 
%                               \begin{description} \item \end{description}
%                               \begin{enumerate} \end{enumerate}}
%                              {\end{cons}}
%\newenvironment{ondorioak}{\begin{cons} \em
%                           \begin{description} \item \end{description}
%                           \begin{enumerate} \end{enumerate}}
%                          {\end{cons}}
\newenvironment{ondoitem}{\begin{ondorioak} \em 
                         \begin{description} \item \end{description}}
                         {\end{ondorioak}}

%\newenvironment{properties}{\begin{ppts}
%                            \begin{description} \item \end{description}}
%                           {\end{ppts}}
\newenvironment{propiitem}{\begin{propietateak} \em 
                           \begin{description} \item \end{description}}
                          {\end{propietateak}}



\bibliographystyle{apalike}
%(apalike eredu bat da, beste asko daude. Elenak google-rekin bilatzen
%ditu).



\begin{document}


\renewcommand{\chaptername}{Gaia}
\renewcommand{\DOCH}{\CNoV \thechapter \space \CNV \FmN{\chaptername} \par\nobreak}

\setcounter{secnumdepth}{3}

%0
%\thispagestyle{empty}
%
%\input{0prev/izenburua}
%
%\newpage
%
%0
%\thispagestyle{empty}
%\cleardoublepage
%
%\pagebreak
%

%1
%\thispagestyle{empty}
%
%\input{0prev/eskerrak}
%
%\newpage
%
%2
%\thispagestyle{empty}
%\cleardoublepage
%
%\pagebreak
%
%3
%\thispagestyle{empty}
%
%\input{0prev/0intro}
%
%\pagebreak
%
%4
%\thispagestyle{empty}
%\cleardoublepage
%
%\pagebreak
%

%\title{ \textbf{Analisi Matematikoa \\ Ariketa ebatziak} }

\title{ \textbf{Kalkulua \\ Ariketa ebatziak} }

\author{Xabier Garrote, Julen Ortiz, Aitor Velaz eta Mikel Alvarez}

\date{\today a}
%\date{2017ko abuztuaren 28a}


\maketitle

\setcounter{page}{1}

\tableofcontents


\vfill{}
%
%\vspace{2mm}
%
%\noindent \textbf{Bibliografía} \dotfill\ \textbf{193}
%
%\vspace{2mm}
%
%\noindent \textbf{Índice alfabético} \dotfill\ \textbf{195}
%
%
\pagebreak

\renewcommand{\thepage}{\arabic{page}}

\setcounter{page}{1}


%%%%%%%%%%%%%%   ANALISI MATEMATIKOA   %%%%%%%%%%%%%%%%%%%%%%%%%%%%
%
%\input{analisia/AMIdatzizkolana/Idatzizkolana}
%
%%%%%%%%%%%%%%   ANALISI MATEMATIKOA   %%%%%%%%%%%%%%%%%%%%%%%%%%%%


%%%%%%%%%%%%%%   KALKULUA   %%%%%%%%%%%%%%%%%%%%%%%%%%%%
%
\chapter{Aldagai anitzeko funtzioak}
\section{Aldagai anitzeko funtzioen jarraitutasuna}
\underline{10.2.ariketa} Aztertu funtzio honen jarraitasuna
$$f(x,y) = \left\{ \begin{array}{cl}
		x\sin \dfrac{1}{y} +y \sin \dfrac{1}{x},  &  x \neq 0 \, \mbox{ eta } \, y \neq 0, \\
                 0,	                                &  x=0 \mbox{ edo } y=0.
		   \end{array} \right. $$
		   
		 
Funtzio bat jatorrian jarraia izateko honako baldintza hau bete behar du:

\begin{equation*}
    \text{Funtzioa jarraia jatorrian}
    \Longleftrightarrow
    \boxed{\lim_{x,y \to (0,0)}f(x,y)=f(0,0)=0}
\end{equation*}
	   
Baldintza hau betetzen den edo ez jakiteko funtzioaren limitea jatorrian kalkulatu behar dugu. Horretarako, honako pausu hauek jarraituko ditugu:
\begin{enumerate}
    \item Dimetsio bakarreko limitea
    \begin{eqnarray*}
        &g(y)&=\lim_{x \to 0}f(x,y)=\lim_{x \to 0} x\sin \dfrac{1}{y} +y \sin \dfrac{1}{x}= \nexists\\
        &h(x)&=\lim_{y \to 0}f(x,y)=\lim_{y \to 0} x\sin \dfrac{1}{y} +y \sin \dfrac{1}{x}= \nexists
    \end{eqnarray*}
    
    \item Limite berrituak
    \begin{eqnarray*}
        &&\lim_{y \to 0}g(y)=\nexists\\
        &&\lim_{x \to 0}h(x)=\nexists
    \end{eqnarray*}
    
    Limite berrituak ez dira existitzen.Hortaz, ezin dugu 1.9.teorema aplikatu eta ez dugu informaziorik lortu. 
    
    Jarraian, norabide bakarreko limiteekin probatuko dugu. Horrela ikusiko dugu norabide ezberdinez jatorrira hurbiltzen bagara ea limitea 0 den.
    \item Norabide bakarreko limiteak \newline
    \newline
    $\displaystyle{ \lim_{ \begin{array}{c} \scriptstyle (x,y) \rightarrow (0,0) \\ \scriptstyle y=mx \\ \end{array} } f(x,y)= \lim_{x \rightarrow 0} x\sin \dfrac{1}{mx} +mx \sin \dfrac{1}{x}=\lim_{x \rightarrow 0} x\cdot\left(\sin \dfrac{1}{mx} +m \sin \dfrac{1}{x}\right)=\nexists } $.
    \newline
    $\displaystyle{ \lim_{ \begin{array}{c} \scriptstyle (x,y) \rightarrow (0,0) \\ \scriptstyle y=\lambda x^2 \\ \end{array} }f(x,y)= \lim_{x \rightarrow 0} x\sin \dfrac{1}{\lambda x^2} +\lambda x^2 \sin \dfrac{1}{x}=\lim_{x \rightarrow 0} x\cdot\left(\sin \dfrac{1}{\lambda x^2} +\lambda x \sin \dfrac{1}{x}\right)=\nexists } $.
    
    Norabide bakarreko limiteak existitzen ez direnez ez digute informazio gehiago eman.
    
    Beraz, azken aukera limiteen definizioa apikatzea da. 
    \item Definizioa aplikatu
    \begin{align*}
    &\lim_{x,y \to (0,0)}f(x,y)=0\\
    &\forall \epsilon >0 \quad \exists \delta(\epsilon)>0 \quad / \quad0<||(x,y)||<\delta \rightarrow ||f(x,y)||<\epsilon  \\
    \\
    &\text{Hau da frogatu behar dugun baldintza}\\
    &\forall \epsilon >0 \quad \exists \delta(\epsilon)>0 \quad / \quad0<||(x,y)||<\delta \rightarrow \left|x\cdot \sin \frac{1}{y}+ y \sin \frac{1}{x}\right|<\epsilon  \\
    &\text{ Hasteko ezkerreko atala hartu eta sinplifikatzen saiatuko gara}\\
    &\left|x\cdot \sin \frac{1}{y}+ y \sin \frac{1}{x}\right|\leq\left| x\right|\cdot\left |\sin\frac{1}{y}\right |+\left|y\right|\cdot\left|\sin\frac{1}{x}\right|\leq \abs{x}+\abs{y}\\
    &\text{Aurreko desberdintzaren eskuineko atalean lortu duguna ikusita,}\\
    &\text{aukerarik honena baturaren norma erabiltzea da} \;\; \| (x,y) \| =\abs{x}+\abs{y}\ \\
    &\| (x,y)\| < \delta \quad \text{bada,} \quad | x | + | y | < \delta \quad \text{dugu}\\
    &\text{eta hortik lortu dugun}\quad\left|x\cdot \sin \frac{1}{y}+ y \sin \frac{1}{x}\right|\leq \abs{x}+\abs{y}\quad \text{desberdintzaz baliatuz}\\
    &\left|x\cdot \sin \frac{1}{y}+ y \sin \frac{1}{x}\right|< \delta\quad \text{izango dugu}\\
    &\text{Ondorioz nahikoa da}\quad \delta < \epsilon \quad \text{izatea }\\
    &\left|x\cdot \sin \frac{1}{y}+ y \sin \frac{1}{x}\right| < \epsilon \quad \text{ere beteko delako}\\
    &\text{Hau guztiarekin limite bikoitza 0 dela frogatu dugu} \quad \boxed{\lim_{x,y \to (0,0)}f(x,y)=0}
    \end{align*}
\end{enumerate}

\begin{align*}
    &\text{Alde batetik,}\lim_{x,y \to (0,0)}f(x,y)=0\quad \text{frogatu dugu, bestetik,}\quad f(0,0)=0 \quad \text{da.}.\\ &\text{Ondorioz} \quad \boxed{\lim_{x,y \to (0,0)}f(x,y)=f(0,0)=0} \quad \text{beteko da eta funtzioa jarraia da jatorrian.}
\end{align*}










\chapter{Aldagai anitzeko funtzioen diferentziagarritasuna}
\section{Berretura-seriezko garapena}
\underline{25.4.ariketa} Kalkulatu zazu funtzioaren Taylorren garapena ematen den puntuaren inguruan
\begin{equation*}
    f(x,y)=cos(x)cos(y), \quad (0,\frac{\pi}{2}) \text{ puntuan eta 3.ordenaraino.}
\end{equation*}
\begin{equation*}
\begin{split}
    &\text{Tayloren garapenaren formula 3.ordeneraino:}\\
    &f(x,y)=f(a,b)+\frac{1}{1!}(D_1f(a,b)(x-a)+D_2f(a,b)(y-b))+\\
    &\frac{1}{2!}(D_{11}f(a,b)(x-a)^2 +2 D_{12}f(a,b)(x-a)(y-b)+D_{22}f(a,b)(y-b)^2)+\\
    &\frac{1}{3|}(D_{111}f(a,b)(x-a)^3+3D_{112}f(a,b)(x-a)^2(y-b)+3D_{122}f(a,b)(x-a)(y-b)^2+\\
    &D_{222}f(a,b)(y-b)^3)\\
\end{split}
\end{equation*}

Lehenengo eta behin, formulan ordezkatu behar ditugun gai guztiak kalkulatuko ditugu.
\begin{equation*}
    \begin{array}{lll}
    & f(x,y)=\cos{x}+\cos{y}\qquad &f(a,b)=f(0,\frac{\pi}{2})=\cos{0}\cos{\frac{\pi}{2}}=0\\
    &(a,b)=(0,\frac{\pi}{2})&\\
    \\
    & D_1f(x,y)=-\sin{x}\cos{y}\qquad & D_1f(0,\frac{\pi}{2})=-\sin{0}\cos{\frac{\pi}{2}}=0 \\
    & D_2f(x,y)=-\cos{x}\sin{y}& D_2f(0,\frac{\pi}{2})=-\cos{0}\sin{\frac{\pi}{2}=-1}\\
    \\
    & D_{11}f(x,y)=-\cos{x}\cos{y}& D_{11}f(0,\frac{\pi}{2})=-\cos(0)\cos{\frac{\pi}{2}}=0\\
    
    & D_{12}f(x,y)=\sin{x}\sin{y} &D_{12}f(0,\frac{\pi}{2})=\sin{0}\sin{\frac{\pi}{2}}=0\\
    
    &D_{22}f(x,y)=-\cos{x}\cos{y}&D_{22}f(0,\frac{\pi}{2})=-\cos{0}\cos{\frac{\pi}{2}}=0\\
    \\
    &D_{111}f(x,y)=-\sin{x}\cos{y}&D_{111}f(0,\frac{\pi}{2})=-\sin{0}\cos{\frac{\pi}{2}}=0\\
    
    &D_{112}f(x,y)=\cos{x}\sin{y}&D_{112}f(0,\frac{\pi}{2})=\cos{0}\sin{\frac{\pi}{2}}=1\\
    
    &D_{122}f(x,y)=\sin{x}\cos{y}&D_{122}f(0,\frac{\pi}{2})=\sin{0}\cos{\frac{\pi}{2}}=0\\
    
    &D_{222}f(x,y)=\cos{x}\sin{y}&D_{222}f(0,\frac{\pi}{2})=\cos{0}\sin{\frac{\pi}{2}}=1\\
    \end{array}
\end{equation*}
Ondoren, 0 atera zaizkigun gai guztiak formulatik ken ditzakegu
\begin{equation*}
\begin{split}
    &f(x,y)=\cancel{f(a,b)}+\frac{1}{1!}(\cancel{D_1f(a,b)(x-a)}+D_2f(a,b)(y-b))+\\
    &\frac{1}{2!}(\cancel{D_{11}f(a,b)(x-a)^2} +2 \cancel{D_{12}f(a,b)(x-a)(y-b)}+\cancel{D_{22}f(a,b)(y-b)^2)}+\\
    &\frac{1}{3|}(\cancel{D_{111}f(a,b)(x-a)^3}+3D_{112}f(a,b)(x-a)^2(y-b)+\cancel{3D_{122}f(a,b)(x-a)(y-b)^2}+\\
    &D_{222}f(a,b)(y-b)^3)\\
    &f(x,y)=D_2f(a,b)\left(y-b\right)+\frac{1}{3!}\left(3D_{112}f(a,b)(x-a)^2(y-b)+ D_{222}f(a,b)(y-b)^3 \right)\\
    &\\
\end{split}
\end{equation*}
Azkenik goian kalkulatutako gaiak formulan ordezkatuko ditugu.
\begin{align*}
    &f(x,y)=D_2f(0,\frac{\pi}{2})\left(y-\frac{\pi}{2}\right)+\frac{1}{3!}\left(3D_{112}f(0,\frac{\pi}{2})(x-0)^2(y-\frac{\pi}{2})+ D_{222}f(0,\frac{\pi}{2})(y-\frac{\pi}{3})^3 \right)=\\
    &-\left(y-\frac{\pi}{2}\right)+\frac{1}{6}\left(3x^2(y-\frac{\pi}{2})+(y-\frac{\pi}{2})^3\right)=-\left(y-\frac{\pi}{2}\right)+\frac{1}{2}x^2\left(y-\frac{\pi}{2}\right)+\frac{1}{6}\left(y-\frac{\pi}{2}\right)^3
\end{align*}


Beraz, ona hemen emaitza:
\begin{equation*}
    \boxed{f(x,y)=-\left(y-\frac{\pi}{2}\right)+\frac{1}{2}x^2\left(y-\frac{\pi}{2}\right)+\frac{1}{6}\left(y-\frac{\pi}{2}\right)^3}
\end{equation*}

\chapter{Aldagai anitzeko funtzioen analisi lokala}
\section{Aldagai anitzeko funtzioen mutur baldintzatuak}
\underline{19.ariketa} Zer luze dira azalera txikiena duen V bolumeneko paralelepipedoaren ertzak? Eta azalera handienekoarenak?

\begin{equation*}
    V = xyz
\end{equation*}
\begin{equation*}
    A(X,Y,Z) = 2xy + 2xz + 2zy = 2(xy+xz+xy)
\end{equation*}
\begin{equation*}
    F(x,y,z) = (2xy + 2xz + 2zy) + \lambda(xyz-V)
\end{equation*}
\newline
\begin{center}
$\left.
  \begin{array}{111}
    & D_{1f}(x,y,z)=2(y+z) + \lambda(yz)& \\
    & D_{2f}(x,y,z)=2(x+z) + \lambda(xz)& \\
    & D_{1f}(x,y,z)=2(x+y) + \lambda(yx)& \\
    & V = xyz 
  \end{array}
  \right\}
  \rightarrow
  \left.
  \begin{array}{111}
    & 2(y+z) + \lambda(yz) = 0& \\
    & 2(x+z) + \lambda(xz) = 0& \\
    & 2(x+y) + \lambda(yx) = 0& \\
    & V = xyz 
  \end{array}
  \right\}
  \rightarrow
    \left.
  \begin{array}{111}
    & \lambda = \frac{-2(y+z)}{yz}& \\
    & \lambda = \frac{-2(x+z)}{xz}& \\
    & \lambda = \frac{-2(x+y)}{yx}& \\
  \end{array}
  \right\}
  \rightarrow
  \left\{
  \begin{array}{111}
    & \frac{-2(y+z)}{yz} = \frac{-2(x+z)}{xz} \rightarrow (y+z)xz=(x+z)yz \rightarrow z(yx+zx-yx-zy)=0&\\
    & \frac{-2(y+z)}{yz} = \frac{-2(x+y)}{xy} \rightarrow (y+z)xy=(x+y)yz \rightarrow xy^2 + xyz = xyz -y^2z& \\
  \end{array}
  \right.
  \rightarrow$
  $\left\{
  \begin{array}{111}
    & z(zx-zy)=0 \rightarrow z^2(x-y)=0 \rightarrow \cancel{z=0} \text{ edo } x=y & \\
    & xy^2 + xyz -xyz +y^2z=0 \rightarrow y^2(x+z)=0 \rightarrow \cancel{y=0} \text{ edo } x=-z & \\
  \end{array}
  \right.
  \rightarrow$
  $\left.
  \begin{array}{111}
    & x=y=-z& \\
    & V=xyz& \\
  \end{array}
  \right\}
  \rightarrow
  P(\sqrt[3]{V}, \sqrt[3]{V}, \sqrt[3]{V})$
\end{center}




\chapter{Integral mugagabea}
\underline{7.27.ariketa} Kalkulatu ondoko integral hau
\begin{equation*}
    \int \frac{\sqrt{x}-\sqrt[6]{x}}{\sqrt[3]{x}+1}\ dx;
\end{equation*}
(TODO: Azaldu)
Integrala berehalakoa ez denez integrazio metodo bat erabili behar dugu, kasu honetan aldagai-aldaketa erabiliko dugu. $t=\sqrt[6]{x}$ eta $dt=dx$ ordezkapenak egingu ditugu.
\begin{equation*}
    \begin{split}
    &\int \frac{\sqrt{x}-\sqrt[6]{x}}{\sqrt[3]{x}+1}\ dx = \int \frac{t^3-t}{t^2+1}6t^5dt = 6 * \int \frac{t^8-t^6}{t^2+1}dt =
     6 * \int \frac{(t^2+1)(t^6-2t^9+2t^2-2)+2}{(t^2+1)}dt = & 
     \\ &6 * \int t^6-2t^4+2t^2-2+\frac{2}{t^2+1}dt = &
     \end{split}
\end{equation*}
Integrala sinplifikatu ondoren berehalakoa bihurtzen da.
\begin{equation*}
     \begin{split}
     &6 * [\frac{t^7}{7}-\frac{2t^5}-{5}+\frac{2t^3}{3}-2t+2\int \frac{1}{t^2+1}dt] = &
     \\ & \frac{6}{7} * t^7 - \frac{12}{5}t^5+4t^3-12t+12\arctan{t} + k = &
     \\ &\frac{6}{7} \sqrt[6]{x^7}-\frac{12}{5}\sqrt[6]{x^5}+4+\sqrt{x}-12*\sqrt[6]{x}+12\arctan{(\sqrt[6]{x})}+k&
    \end{split}
\end{equation*}




\chapter{Integral mugatua}
\section{Integral mugatuen aplikazioak}
\underline{13.7.ariketa} Kalkulatu ondoko funtzioak osatzen duen gorputzaren bolumena

\begin{equation*}
    x^2+(y-b)^2=a^2,\ \text{torua}\ b>a \ \text{izanik;}
\end{equation*}
%
%%%%%%%%%%%%%%   KALKULUA   %%%%%%%%%%%%%%%%%%%%%%%%%%%%

\printindex

\end{document}
